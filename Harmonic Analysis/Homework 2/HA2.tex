\documentclass[12 pt]{article}
\usepackage[utf8]{inputenc} 
\usepackage[T1,T2A]{fontenc}
\usepackage[english, ukrainian]{babel}
\usepackage{geometry}
\usepackage{amsmath}
\usepackage{mathabx}
\usepackage{bbm}
\usepackage{amsfonts}
\usepackage{graphicx}
\usepackage{pgfplots}
\usepackage{adjustbox}
\usepackage{cancel}
\geometry{verbose,a4paper,tmargin=2cm,bmargin=2cm,lmargin=1cm,rmargin=1cm}
\linespread{1.25}

\begin{document}
\begin{center}
    \Large
    Романович Володимир Володимирович КА-02 \\
    Самостійна робота №2 \\ 
    Варіант 9
\end{center}
\begin{center}
    \Large 
    Задача 1
\end{center}
$\mathcal{S} = \{ [a;b] : a, b \in \mathbb{R} \setminus \mathbb{Q}, a < b \}$ \\
Бачимо, що оскільки a < b, то $\emptyset \notin \mathcal{S}$, тому $\mathcal{S}$ -- не кільце \\ 
Розглянемо тепер $A = \left[\sqrt{2}, \sqrt{5} \right] \in \mathcal{S}, 
\ B = \left[\sqrt{3}, \sqrt{6} \right] \in \mathcal{S}$ \\ 
Тоді $A \setminus B = \left[\sqrt{2}; \sqrt{3}\right)$. \\
Очевидно, що неможливо
підібрати таке $\bigsqcup_{k=1}^{n} 
[a_{k}; b_{k}], a_k, b_k \in \mathbb{R} \setminus \mathbb{Q}, a_k < b_k$, 
щоб отримати $[\sqrt{2};\sqrt{3})$, як мінімум тому, що 
об'єднання відрізків -- замкнута множина на $\mathbb{R}$ , а $[\sqrt{2};\sqrt{3})$ -- відкрита. \\ 
Отже $\mathcal{S}$ -- не півкільце. \\ 
Відповідь: $\mathcal{S}$ не є ні кільцем, ні півкільцем.
\begin{center}
    \Large 
    Задача 2
\end{center}
Побудувати кільце та $\sigma$-кільце, породжені сім'єю S підмножин множини $\Omega$ \\ 
$S = \{ (0;\frac{1}{n}) : n \in \mathbb{N}  \}, \ \Omega = (0;1) $ 
\\
Розглянемо кільце 
$R_s = \left\{ (0;\frac{1}{n}), 
\bigcup_{i=1}^m\limits \left[\frac{1}{p_i};\frac{1}{k_i}\right), 
(0;\frac{1}{n}) \cup  
\bigcup_{i=1}^m\limits \left[\frac{1}{p_i};\frac{1}{k_i}\right) :
n, p_i, k_i \in \mathbb{N}, \ k_i \leq p_i  \right\}$ \\ 
$S \subset R_s$. Доведемо, що $R_s$ -- найменше кільце, яке містить S: \\ 
Нехай R -- інше кільце, що містить S. Тоді: \\
1) $\emptyset \in R$ \\
2) $\left(0;\frac{1}{n_1}\right) \setminus \left(0;\frac{1}{n_1} \right) = \begin{cases}
    \emptyset, & n_2 \leq n_1 \\ 
    [\frac{1}{n_2}; \frac{1}{n_1}), & n_2 > n_1 
\end{cases} \in R \implies \left[\frac{1}{n_2}; \frac{1}{n_1}\right) \in R, \quad n_1, n_2 \in \mathbb{N}, n_2 > n_1 $ \\ 
3) $\bigcup_{i=1}^m\limits  \left(0;\frac{1}{n_i}\right) = \left(0; \frac{1}{n}\right) \in R$, \
$\bigcup_{i=1}^m\limits \left[\frac{1}{p_i}; \frac{1}{k_i}\right) \in R$, \ $
\left(\bigcup_{j=1}^v\limits \left(0;\frac{1}{n_j}\right)\right) \cup \left(\bigcup_{i=1}^m\limits \left[\frac{1}{p_i}; \frac{1}{k_i}\right)\right) = 
(0;\frac{1}{n}) \cup \left(\bigcup_{i=1}^m\limits \left[\frac{1}{p_i}; \frac{1}{k_i}\right)\right) \in R$ 
Отже, якщо R кільце та воно містить S, то \\ 
$R \supset \left\{ \emptyset, (0;\frac{1}{n}), 
\bigcup_{i=1}^m\limits \left[\frac{1}{p_i};\frac{1}{k_i}\right), 
(0;\frac{1}{n}) \cup  
\bigcup_{i=1}^m\limits \left[\frac{1}{p_i};\frac{1}{k_i}\right) :
n, p_i, k_i \in \mathbb{N}, \ k_i < p_i  \right\}$ \\ 
Оскільки $\emptyset$ можна записати як $\left[\frac{1}{p_i};\frac{1}{k_i}\right)$, $p_i=k_i$,  то $R_s \subset R$, а отже
є кільцем, породженим сім'єю S.   \\ \\ 
Розглянемо $\sigma$-кільце 
$K_s = \left\{ (0;\frac{1}{n}), 
\bigcup_{i=1}^{\infty}\limits \left[\frac{1}{p_i};\frac{1}{k_i}\right), 
(0;\frac{1}{n}) \cup  
\bigcup_{i=1}^{ \infty }\limits \left[\frac{1}{p_i};\frac{1}{k_i}\right) :
n, p_i, k_i \in \mathbb{N}, \ k_i \leq p_i \ \right\}$ \\ 
$S \subset K_s$. Доведемо, що $K_s$ -- найменше $\sigma$-кільце, яке містить S: \\
Нехай K -- інше $\sigma$-кільце, та воно містить S. $\sigma$-кільце, що містить S 
містить всі елементи кільця, породженого сім'єю S та замкнуте відносно зліченного об'єднання, тому \\ 
1)$\bigcup_{k \in \mathbb{N}}\limits (0; \frac{1}{n_k}) = (0; \frac{1}{n}) \in K$ \\ 
2)$\bigcup_{k \in \mathbb{N}}\limits \left[\frac{1}{p_i};\frac{1}{k_i}\right)\in K, k_i \leq p_i$ \\ 
3)$(0;\frac{1}{n}) \cup \bigcup_{k \in \mathbb{N}}\limits \left[\frac{1}{p_i};\frac{1}{k_i}\right)\in K, k_i \leq p_i$ \\ 
Отримуємо, що $K_s \subset K$, тому $K_s$ -- $\sigma$-кільце, породжене сім'єю S. 


\begin{center}
    \Large 
    Задача 3
\end{center}
$A = (-1;1) \cap \{ x \in \mathbb{R} : \ln{(1+x)} \in \mathbb{Q} \}= \{ x \in (-1;1) : \ln{(1+x)} \in \mathbb{Q} \}$ \\ 
Розглянемо $A_q$  = $\{ x \in (-1;1) : \ln{(1+x)} = q \}$ \\ 
$\ln{(1+x)} = q \implies x= e^q - 1$, \ $-1 < x < 1 \implies -1 < e^q -1 < 1 \implies q < \ln2$ \\  
Отже $A_q = \{ x \in \mathbb{R} : x = e^q-1, q \in \mathbb{Q}\setminus[\ln2; +\infty ) \}$ \\ 
Позначимо $\mathbb{Q}_A = \mathbb{Q} \setminus [\ln2; +\infty )$ \\ 
Оскільки $\mathbb{Q}$ -- зліченна множина, то 
$\mathbb{Q}_A \subset \mathbb{Q}$ -- також зліченна \\ 
$A_q$ -- одноточкова множина, тому $A_q \in \mathcal{B}$ і $\lambda(A_q) = 0$  \\ 
$A = \bigsqcup_{q \in \mathbb{Q}_A}\limits A_q, \ A \in \mathcal{B} $,
 оскільки $\mathcal{B} - \sigma\text{-кільце}$. \\ 
$\lambda(A) = \lambda\left(\bigsqcup_{q \in \mathbb{Q}_A}\limits A_q\right) = 
\sum_{q \in \mathbb{Q}_A}\limits\lambda(A_q) = \sum_{q \in \mathbb{Q}_A}\limits0 = 0$ 

\begin{center}
    \Large 
    Задача 4
\end{center}
Довести, що функція $f: \mathbb{R}^2 \to \mathbb{R}$ -- вимірна. \\ 
$f(x,y) = [x^2-y^2] \cos{\pi x y}$ \\ 
Розглянемо функції $g_1 : \mathbb{R}^2 \to \mathbb{R}, g_2 : \mathbb{R}^2 \to \mathbb{R}, g_3: \mathbb{R}\to \mathbb{R}$: \\ 
$g_1 (x,y) = x^2 - y^2$ -- неперервна на $\mathbb{R}^2$ , тому вимірна за теоремою. \\ 
$g_2 (x,y) = \cos{\pi x y}$  -- неперервна на $\mathbb{R}^2$ , тому вимірна за теоремою. \\ 
$g_3 (x) = [x]$ \\ 
Розглянему множину $\{ f \leq a \}$ 


	\begin{tikzpicture}[scale=1]
        \fill[red!10] (-4, -4) -- (-4, 2.2) -- (4, 2.2) -- (4, -4);
        \draw[thick, red] (-4, 2.2) -- (4, 2.2);
		\draw[->] (-4,0) -- (4,0) node[below] {$x$};
		\draw[->] (0,-4) -- (0,4) node[left] {$g_3(x)$};
        
        \draw[ultra thick, black] (-2, -3) -- (-2.95, -3);
        \draw[ultra thick, black] (-1, -2) -- (-1.95, -2);
        \draw[ultra thick, black] (0, -1) -- (-0.95, -1);
        \draw[ultra thick, black] (0, 0) -- (0.95, 0);
        \draw[ultra thick, black] (1, 1) -- (1.95, 1);
        \draw[ultra thick, black] (2, 2) -- (2.95, 2);
        \draw[ultra thick, black] (3, 3) -- (3.95, 3);
        \draw (1.98,1) circle[radius=1.5pt];
        \draw (2.98,2) circle[radius=1.5pt];
        \draw (3.98,3) circle[radius=1.5pt];
        \draw (0.98,0) circle[radius=1.5pt];
        \fill (1,1) circle[radius=2pt];
        \fill (2,2) circle[radius=2pt];
        \fill (3,3) circle[radius=2pt];
        \fill (3,3) circle[radius=2pt];
        \fill (0,0) circle[radius=2pt];
        \draw (0,-1) circle[radius=1.5pt];
        \draw (-0.98,-2) circle[radius=1.5pt];
        \draw (-1.98,-3) circle[radius=1.5pt];
        \fill (-2,-2) circle[radius=2pt];
        \fill (-3,-3) circle[radius=2pt];
        \fill (-1,-1) circle[radius=2pt];
        
        \draw (-3.8, 2.2) node[above] {a};

        \draw (3, 0) node[below] {3};
        \draw (2, 0) node[below] {2};
        \draw (1, 0) node[below] {1};
        \draw (0, 2) node[left] {2};
        \draw (0, 1) node[left] {1};
		\draw (0, 0) node[below left] {0};
        \draw (0, -1) node[right] {-1};
	\end{tikzpicture} \\ 
$\forall a \in \mathbb{R}: [x] \leq a \implies x \in (-\infty; [a] + 1) \in \mathcal{B}$
як відкрита півпряма, отже за означенням $g_3(x)$ -- вимірна. \\
Тоді $[x^2 - y^2] = g_3 \circ g_1 $ -- вимірна за твердженням, як композиція вимірних функцій. \\ 
Отже $f(x,y) = [x^2-y^2] \cdot \cos{\pi x y}$ -- вимірна за теоремою, як добуток вимірних функцій. 
\newpage
\begin{center}
    \Large 
    Задача 5
\end{center}
$f(x) = [x] sgn(\sin{\frac{\pi}{2}x}), A=[-3,2]$ \\ \\ 
\begin{tikzpicture}[scale=0.9]
    \draw[->] (-3.5,0) -- (2.5,0) node[below] {$x$};
    \draw[->] (0,-3.5) -- (0,3.5) node[left] {$f(x)$};

    \draw[ultra thick, black] (-3, -3) -- (-2.05, -3);
    \draw[ultra thick, black] (-2, 2) -- (-1.05, 2);
    \draw[ultra thick, black] (-1, 1) -- (-0.05, 1);
    \draw[ultra thick, black] (0, 0) -- (0.95, 0);
    \draw[ultra thick, black] (1, 1) -- (1.95, 1);

    \draw (-2.02, -3) circle[radius=1.5pt];
    \draw (-2.02, 2) circle[radius=1.5pt];
    \draw (-1.02, 2) circle[radius=1.5pt];
    \draw (-0.02, 1) circle[radius=1.5pt];
    \draw (0.98, 0) circle[radius=1.5pt];
    \draw (1.98, 1) circle[radius=1.5pt];
    \fill (-3,-3) circle[radius=2pt];
    \fill (-2,0) circle[radius=2pt];
    \fill (-1,1) circle[radius=2pt];
    \fill (0,0) circle[radius=2pt];
    \fill (1,1) circle[radius=2pt];
    \fill (2,0) circle[radius=2pt];
    \draw (2, 0) node[below] {2};
    \draw (1, 0) node[below] {1};
    \draw (-1, 0) node[below] {-1};
    \draw (-2, 0) node[below] {-2};
    \draw (0, 2) node[right] {2};
    \draw (0, 1) node[right] {1};
    \draw (0, 0) node[below left] {0};
    \draw (0, -1) node[right] {-1};
    \draw (0, -3) node[right] {-3};
\end{tikzpicture} \qquad
\begin{tikzpicture}[scale=0.9]
    \draw[->] (-3.5,0) -- (2.5,0) node[below] {$x$};
    \draw[->] (0,-3.5) -- (0,3.5) node[left] {$f^{+}(x)$};

    \draw[ultra thick, black] (-3, 0) -- (-2.05, 0);
    \draw[ultra thick, black] (-2, 2) -- (-1.05, 2);
    \draw[ultra thick, black] (-1, 1) -- (-0.05, 1);
    \draw[ultra thick, black] (0, 0) -- (0.95, 0);
    \draw[ultra thick, black] (1, 1) -- (1.95, 1);

    \draw (-2.02, 2) circle[radius=1.5pt];
    \draw (-1.02, 2) circle[radius=1.5pt];
    \draw (-0.02, 1) circle[radius=1.5pt];
    \draw (0.98, 0) circle[radius=1.5pt];
    \draw (1.98, 1) circle[radius=1.5pt];
    \fill (-2,0) circle[radius=2pt];
    \fill (-3,0) circle[radius=2pt];
    \fill (-1,1) circle[radius=2pt];
    \fill (0,0) circle[radius=2pt];
    \fill (1,1) circle[radius=2pt];
    \fill (2,0) circle[radius=2pt];
    \draw (2, 0) node[below] {2};
    \draw (1, 0) node[below] {1};
    \draw (-1, 0) node[below] {-1};
    \draw (-2, 0) node[below] {-2};
    \draw (0, 2) node[right] {2};
    \draw (0, 1) node[right] {1};
    \draw (0, 0) node[below left] {0};
    \draw (0, -1) node[right] {-1};
    \draw (0, -3) node[right] {-3};
\end{tikzpicture}\qquad 
\begin{tikzpicture}[scale=0.9]
    \draw[->] (-3.5,0) -- (2.5,0) node[below] {$x$};
    \draw[->] (0,-3.5) -- (0,3.5) node[left] {$f^{-}(x)$};

    \draw[ultra thick, black] (-3, 3) -- (-2.05, 3);
    \draw[ultra thick, black] (-2, 0) -- (2, 0);
    \draw (-2.02, 3) circle[radius=1.5pt];

    \fill (-3,3) circle[radius=2pt];
    \fill (-2,0) circle[radius=2pt];
    \fill (2,0) circle[radius=2pt];
    \draw (2, 0) node[below] {2};
    \draw (1, 0) node[below] {1};
    \draw (-1, 0) node[below] {-1};
    \draw (-2, 0) node[below] {-2};
    \draw (0, 2) node[right] {2};
    \draw (0, 1) node[right] {1};
    \draw (0, 0) node[below left] {0};
    \draw (0, -1) node[right] {-1};
    \draw (0, -3) node[right] {-3};
\end{tikzpicture} \\
Позначимо:
$A_1 = [-3; -2], A_2 = (-2; -1), A_3 = [-1;0), A_4 = [0;1), A_5 = [1;2), A_6 = \{2\}$  \\ 
$B_1 = [-3;-2), B_2 = [-2;2]$ \\ 
Тоді: \\ 
$f^+(x) = \begin{cases}
    0, & x \in A_1 \cup A_4 \cup A_6 \\
    1, & x \in A_3\cup A_5 \\ 
    2, & x \in A_2\\ 
\end{cases}, \quad 
f^-(x) = \begin{cases}
    0, & x \in B_2  \\
    3, & x \in B_1  \\ 
\end{cases}$ \\ 
За означенням: 
$$
\int_A f^+ d \lambda = \int_{[-3;2]} f^+(x) d \lambda(x) = \sum_{k=1}^m c_k \lambda(A_k)  = 
$$
$$
=0 \cdot \lambda([-3;-2]) + 2 \cdot \lambda((-2;-1)) + 
1 \cdot \lambda([-1;0)) + 0 \cdot \lambda([0;1)) + 1 \cdot\lambda([1;2))
+ 0 \cdot \lambda(\{2\})  = 4
$$
$$
\int_A f^- d \lambda = \int_{[-3;2]} f^-(x) d \lambda(x) = \sum_{k=1}^m c_k \lambda(B_k)  = 
3 \cdot \lambda([-3;-2)) + 0 \cdot \lambda([-2;2]) = 3
$$
$$
\int_A |f| d \lambda = \int_A f^+ d \lambda + \int_A f^- d \lambda = 4 + 3 = 7
$$
$$
\int_A f d \lambda = \int_A f^+ d \lambda - \int_A f^- d \lambda = 4 - 3 = 1
$$

\begin{center}
    \Large 
    Задача 6
\end{center}
Знайти $\lim_{n \to \infty}\limits \int_{A} f_n(x) \,d \lambda(x)  $ \\
$f_n(x) = e^{-x}(1+\cos^n{x}), A = \mathbb{R}_+ $ \\ 
$$
\lim_{n \to \infty} f_n(x) =  
\lim_{n \to \infty} e^{-x}(1+\cos^n{x}) =  \begin{cases}
    e^{-x}(1+0), & |\cos{x}| < 1 \\ 
    e^{-x}(1+1), & \cos{x} = 1 \\ 
    e^{-x}(1+(-1)^n), & \cos{x} = - 1 \\ 
\end{cases} 
$$
$\tilde A = \{x \in A : \cos{x} = \pm 1\}$ -- зліченна множина, тому $\lambda(\tilde A) = 0$ \\ 
Отже $f_n(x) \to e^{-x} \ \lambda$-майже всюди, $n \to \infty$ \\ 
Розглянемо $\int_{\mathbb{R}_+} e^{-x} \,d \lambda(x) $: \\ 
Нехай $g_n(x) = e^{-x} \cdot \mathbbm{1}_{[0;n]}(x)$, тоді $g_n(x) \uparrow e^{-x} \  \forall x \in \mathbb{R}$ \\ 
Тоді за наслідком з теореми Беппо-Леві:
$$
\int_{\mathbb{R}_+} e^{-x} \,d \lambda(x) = 
\lim_{n \to \infty}  \int_{\mathbb{R}_+} g_n(x) \,d \lambda(x) = 
\lim_{n \to \infty}  \int_{\mathbb{R}_+} e^{-x} \cdot \mathbbm{1}_{[0;n]}(x)  \,d \lambda(x) = 
$$
$$=
\lim_{n \to \infty}  \int_{[0;n]} e^{-x} \,d \lambda(x)
\stackrel{(1)}{=} \lim_{n \to \infty}  \int_0^n e^{-x} \,dx =
\lim_{n \to \infty}  (1-e^{-n}) = 1 < \infty
$$
Перехід (1) здійснюється за теоремою, завдяки тому, що $e^{-x}$ -- інтегровна за Ріманом. \\ 
Отримали що $e^{-x} \in \mathcal{L}^1(\mathbb{R}_+, \lambda ) $ та $\int_{\mathbb{R}_+} e^{-x} \,d \lambda(x) = 1$ \\ 
Використаємо теорему Лебега про обмежену збіжність: \\ 
1) $f_n(x) \to e^{-x} \  \lambda$-майже всюди, $n \to \infty$  \\ 
2) $\forall x \in \mathbb{R}_+ : |f_n(x)| = |e^{-x}(1+\cos^n{x})| 
\leq |e^{-x} \cdot 2| = 2 e^{-x} = g(x), \ \ g(x) \in \mathcal{L}^1(\mathbb{R}_+, \lambda )$ \\ 
Тому $\lim_{n \to \infty} \limits \int_{\mathbb{R}_+} f_n(x) \,d \lambda(x) = 
\int_{\mathbb{R}_+} e^{-x} \,d \lambda(x) = 1$ 
\end{document}