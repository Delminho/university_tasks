\documentclass[12 pt]{article}
\usepackage[utf8]{inputenc} 
\usepackage[T1,T2A]{fontenc}
\usepackage[english, ukrainian]{babel}
\usepackage{geometry}
\usepackage{amsmath}
\usepackage{mathabx}
\usepackage{amsfonts}
\usepackage{cancel}
\geometry{verbose,a4paper,tmargin=2cm,bmargin=2cm,lmargin=1cm,rmargin=1cm}
\linespread{1.25}

\begin{document}
\begin{center}
    \Large
    Романович Володимир Володимирович КА-02 \\
    Самостійна робота №2 \\ 
    Варіант 9
\end{center}
\begin{center}
    \Large 
    Задача 1
\end{center}
$\mathcal{S} = \{ [a;b] : a, b \in \mathbb{R} \setminus \mathbb{Q}, a < b \}$ \\
Бачимо, що оскільки a < b, то $\emptyset \notin \mathcal{S}$, тому $\mathcal{S}$ -- не кільце \\ 
Розглянемо тепер $A = \left[\sqrt{2}, \sqrt{5} \right] \in \mathcal{S}, 
\ B = \left[\sqrt{3}, \sqrt{6} \right] \in \mathcal{S}$ \\ 
Тоді $B \setminus A = \left[\sqrt{2}; \sqrt{3}\right)$. \\
Очевидно, що неможливо
підібрати таке $\bigsqcup_{k=1}^{n} 
[a_{k}; b_{k}], a_k, b_k \in \mathbb{R} \setminus \mathbb{Q}, a_k < b_k$, 
щоб отримати $[\sqrt{2};\sqrt{3})$, як мінімум тому, що 
об'єднання відрізків -- замкнута множина на $\mathbb{R}$ , а $[\sqrt{2};\sqrt{3})$ -- відкрита. \\ 
Отже $\mathcal{S}$ -- не півкільце. \\ 
Відповідь: $\mathcal{S}$ не є ні кільцем, ні півкільцем.
\begin{center}
    \Large 
    Задача 2
\end{center}

\end{document}