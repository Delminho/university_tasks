\documentclass[14 pt]{extarticle}
\usepackage[utf8]{inputenc} 
\usepackage[T1,T2A]{fontenc}
\usepackage[english, ukrainian]{babel}
\usepackage[center]{titlesec}
\usepackage{amsmath, amsfonts}
\usepackage{mathtools}
\usepackage{marvosym}
\usepackage{geometry}
\geometry{verbose,a4paper,tmargin=1.5cm,bmargin=1.5cm,lmargin=1.8cm,rmargin=2cm}
\usepackage{diagbox}
\usepackage{hyperref}
\usepackage{nicefrac}
\usepackage{graphicx}
\usepackage{pgfplots}
\usepackage{physics}
\usepackage{relsize}
\usepackage{textgreek}
\usepackage{upgreek}
\usepackage{adjustbox}
\linespread{1.25}
\makeatletter
\renewcommand*\env@matrix[1][\arraystretch]{%
  \edef\arraystretch{#1}%
  \hskip -\arraycolsep
  \let\@ifnextchar\new@ifnextchar
  \array{*\c@MaxMatrixCols c}}
\makeatother

\usepgfplotslibrary{fillbetween}
\usepackage{float}
\hypersetup{
    colorlinks=true,
    linkcolor = blue,
}
\DeclareMathOperator*{\argmax}{argmax} % thin space, limits underneath in displays
\usepackage{tikz}
\newcommand*\circled[1]{\tikz[baseline=(char.base)]{
            \node[shape=circle,draw,inner sep=2pt] (char) {#1};}}

            


\begin{document}



\begin{titlepage}
    \begin{center}
        
        \includegraphics[width=18cm]{image1.png}
        Мiнiстерство освiти i науки України \\
        Національний технічний університет України \\
        «Київський політехнічний інститут імені Ігоря Сікорського» Iнститут прикладного системного аналiзу \\

        \vspace{4.5cm}
        \textbf{Лабораторна робота № 2}\\
        з курсу «Чисельнi методи» \\
        з теми «Ітераційні методи розв'язвння СЛАР» \\
        Варіант № 9 \\
    \end{center}

    \vspace{2cm}

    \begin{flushright}
        Виконав студент 2 курсу групи КА-02 \\
        Романович Володимир Володимирович \\
        перевiрила старший викладач \\
        Хоменко Ольга Володимирiвна \\

    \end{flushright}

    \vfill
    \begin{center}
        Київ -- 2022
    \end{center}
\end{titlepage}

\begin{center}
    \large
    \textbf{Завдання}
\end{center}
\small{
    Методом Якобі або Зейделя розв'язати систему рівнянь:\\ 
    $$
    \begin{cases}
        2.389 x_1 + 0.273 x_2 + 0.126 x_3 + 0.418 x_4 = 0.144 \\ 
        0.329 x_1 + 2.796 x_2 + 0.179 x_3 + 0.278 x_4 = 0.297 \\ 
        0.186 x_1 + 0.275 x_2 + 2.987 x_3 + 0.316 x_4 = 0.529 \\ 
        0.197 x_1 + 0.219 x_2 + 0.274 x_3 + 3.127 x_4 = 0.869 
    \end{cases}
    $$
    Маємо матрицю А = $
    \begin{pmatrix}
        2.389 & 0.273 & 0.126 & 0.418 \\ 
        0.329 & 2.796 & 0.179 & 0.278  \\ 
        0.186 & 0.275 & 2.987 & 0.316  \\ 
        0.197 & 0.219 & 0.274 & 3.127 
    \end{pmatrix}$\\ 
    Перевіримо її на діагональну перевагу: \\ 
    $
    2.389 > 0.817 = 0.273 + 0.126 + 0.418  \\ 
    2.796 > 0.786 = 0.329 + 0.179 + 0.278  \\ 
    2.987 > 0.777 = 0.275 +  0.186+ 0.316  \\ 
    3.127 > 0.69 = 0.219 + 0.274 +  0.197
    $\\ 
    Отже умова діагональної переваги виконується, тому за теоремою метод Зейделя для цієї системи є збіжним,
    причому швидше ніж метод Якобі, тому використовуватимемо саме його. \\ 
    Запишемо систему у вигляді $\vec{x} = (-D^{-1} (L + R) )\vec{x} + D^{-1}\vec{c}$ :
    $$
    \vec{x} = \begin{pmatrix}[1.3]
        0 & -\frac{0.273}{2.389} & -\frac{0.126}{2.389} & -\frac{0.418}{2.389} \\ 
        -\frac{0.329}{2.796} & 0 & -\frac{0.179}{2.796} & -\frac{0.278}{2.796}  \\ 
        -\frac{0.186}{2.987} & -\frac{0.275}{2.987} & 0 & -\frac{0.316}{2.987}  \\ 
        -\frac{0.197}{3.127 } & -\frac{0.219}{3.127 } & -\frac{0.274}{3.127 } & 0 
    \end{pmatrix}
    \cdot \vec{x} 
    + 
    \begin{pmatrix}[1.3]
        \frac{1}{2.389} & 0 & 0 & 0 \\ 
        0 & \frac{1}{2.796} & 0 & 0 \\ 
        0 &  0 & \frac{1}{2.987} & 0 \\ 
        0 & 0 & 0 &\frac{1}{3.127 }
    \end{pmatrix}
    \cdot 
    \begin{pmatrix}
        0.144 \\ 
        0.297 \\ 
        0.529 \\ 
        0.869
    \end{pmatrix}
    $$
    Знайдемо $\|B\|_{ \infty } = \|-D^{-1} (L + R)\|_{ \infty} = 
    \max{\left\{ \frac{817}{2389}, \frac{786}{2796}, \frac{777}{2987}, \frac{690}{3127} \right\}} 
    = \frac{817}{2389} \approx 0.34 < 0.5 $ \\
    Тому критерієм зупинки можемо взяти умову $\| x_* - x^{(k)} \|_{ \infty } \leq \varepsilon = 0.00001 $ \\ 
    Реалізувавши метод Зейделя в Python отримуємо таку відповідь: 
    $\vec{x} \approx \begin{pmatrix}
        -0.000983 \\ 
        0.071286 \\ 
        0.143047 \\
        0.260437
    \end{pmatrix}$  
    
    \begin{table}[H]
        \centering
        \small
        \begin{tabular}{|l|l|l|l|l|l|}
        \hline
            Iteration & $x_1$ & $x_2$ & $x_3$ & $x_4$  & $||x^{(k+1)} - x^{(k)}||$ \\ \hline
            0 & 0.060276 & 0.106223 & 0.177101 & 0.277902 & ~ \\ \hline
            1 & -0.009827 & 0.068410 & 0.142015 & 0.261286 & 0.070103 \\ \hline
            2 & -0.000748 & 0.071240 & 0.142947 & 0.260434 & 0.009079 \\ \hline
            3 & -0.000972 & 0.071292 & 0.143046 & 0.260436 & 0.000224 \\ \hline
            4 & -0.000983 & 0.071286 & 0.143047 & 0.260437 & 0.000011 \\ \hline
            5 & -0.000983 & 0.071286 & 0.143047 & 0.260437 & 0.000000 \\ \hline
        \end{tabular}
        \scriptsize
        \caption{Результат роботи алгоритму з початковим наближенням $D^{-1}\vec{c}$ }
    \end{table}
    \ \\
    Також в програмі знайдемо розв'язок за допомогою функції linalg.solve() з бібліотеки Numpy,
    обчислимо вектор нев'язки $\vec{b} - A \cdot \vec{x}_*$, та застосуємо цей метод з іншим початковим наближенням. \\ 
    Вихідні дані програми: \\ 
    \footnotesize	{
    Our solution: [-0.0009827818268132518, 0.07128626360839627, 0.14304684446284438, 0.26043718609108546] \\
    Numpy function's solution: [-0.00098275  0.07128627  0.14304684  0.26043718] \\ 
    b-A*x:  ['0.000000', '0.000000', '-0.000000', '-0.000000'] \\ 
    Our solution with (10, 20, 30, 40) starting vector: [-0.0009826903538848741, 0.07128627985793794, 0.1430468342637694, 0.2604371800839636] \\
    }
    Бачимо що наш розв'язок $\vec{x}_* \approx \begin{pmatrix}
        -0.000983 \\ 
        0.071286 \\ 
        0.143047 \\
        0.260437
    \end{pmatrix}$  
    дорівнює розв'язку Numpy $\tilde{\vec{x}} = \begin{pmatrix} 
        -0.00098275 \\ 
        0.07128627 \\ 
        0.14304684 \\
        0.26043718
    \end{pmatrix}
    $ з точністю $\varepsilon$ \\ 
    \small
    Також бачимо що вектор нев'язки $\vec{b} - A \cdot \vec{x}_* \approx \vec{0}$ \\ 
    Взявши за початкове наближення вектор $\vec{a} = (10,20,30,40)^T$ отримаємо таку ж відповідь 
    з точністю до $\varepsilon$, але кількість ітерацій збільшилася  
    \begin{table}[H]
        \centering
        \small
        \begin{tabular}{|l|l|l|l|l|l|}
        \hline
            Iteration & $x_1$ & $x_2$ & $x_3$ & $x_4$ & $||x^{(k+1)} - x^{(k)}||$ \\ \hline
            0 & 0.060276 & 0.106223 & 0.177101 & 0.277902 & ~ \\ \hline
            1 & -10.806195 & -4.519943 & -2.965538 & 1.535096 & 38.464904 \\ \hline
            2 & 0.464601 & 0.088777 & -0.022404 & 0.244378 & 11.270796 \\ \hline
            3 & 0.008555 & 0.082353 & 0.143133 & 0.259054 & 0.456047 \\ \hline
            4 & -0.002010 & 0.071539 & 0.143234 & 0.260468 & 0.010814 \\ \hline
            5 & -0.001027 & 0.071276 & 0.143047 & 0.260441 & 0.000983 \\ \hline
            6 & -0.000982 & 0.071286 & 0.143046 & 0.260437 & 0.000045 \\ \hline
            7 & -0.000983 & 0.071286 & 0.143047 & 0.260437 & 0.000000 \\ \hline
        \end{tabular}
        \scriptsize
        \caption{Результат роботи алгоритму з початковим наближенням $\vec{a}$ }
    \end{table}
}
\begin{center}
    \large
    \textbf{Висновок}
\end{center}
\small{
    Переконавшись у виконанні достатньої умови збіжності для нашої системи, ми використали
    метод Зейделя для розв'язання системи із заданою точністю. Отримали відповідь
    $\vec{x} = (-0.00098, 0.07128, 0.14304, 0.26437)^T$, знайшли вектор нев'язки і
    він дорівнює $\vec{0}$ із заданою точністю, тому можемо сказати що отримана відповідь є правильною. 
    Також ми застосували алгоритм вже з іншим початковим наближенням, і, як і очікувалось отримали 
    таку ж відповідь(з точністю до $\varepsilon$), але знадобилось на 2 ітерації більше. Такий
    результат зрозумілий, оскільки якщо метод Зейделя для системи збіжний, то він 
    збіжний при будь якому скінченному початковому наближенню.   
}
\begin{center}
    \large
    \textbf{Лістинг програми}
\end{center}
\scriptsize
\begin{verbatim}
import numpy
import pandas
import openpyxl


def chebyshev_norm(vector):
    """Returns Chebyshev's norm of vector"""
    abs_vector = [abs(val) for val in vector]
    return max(abs_vector)


def vector_subtract(vec1, vec2):
    """Returns vector that is result of subtracting 2 vectors"""
    vec3 = []
    for i in range(len(vec1)):
        vec3.append(vec1[i] - vec2[i])
    return vec3


A_matrix = numpy.array([        # from A*x = b
    [2.389, 0.273, 0.126, 0.418],
    [0.329, 2.796, 0.179, 0.278],
    [0.186, 0.275, 2.987, 0.316],
    [0.197, 0.219, 0.274, 3.127]
])
b_vector = numpy.array([0.144, 0.297, 0.529, 0.869])    # from A*x = b
B_matrix = [        # from x = Bx + c
     [-1/2.389 * val for val in [0, 0.273, 0.126, 0.418]],
     [-1/2.796 * val for val in [0.329, 0,  0.179, 0.278]],
     [-1/2.987 * val for val in [0.186, 0.275, 0, 0.316]],
     [-1/3.127 * val for val in [0.197, 0.219, 0.274, 0]]]
c_vector = [0.144/2.389, 0.297/2.796, 0.529/2.987, 0.869/3.127]     # from x = Bx + c


def zeydel_method(B, c, norm, epsilon, starting_val=None):
    """Solve system of linear equations x = B*x + c using Zeydel method"""
    if not starting_val:
        # Setting vector c as starting value
        prev = list(c)
    else:
        prev = list(starting_val)
    x = list(prev)
    n = len(x)
    # Variables for displaying information in table
    iteration = 1
    # Array that holds information for displaying in table
    info_arr = [[0] + list(c)]
    while True:

        for i in range(n):
            x_i = 0
            # Finding b_i0 * x_0 + b_i1 * x_1 + ... + b_in * x_n sum and storing in x_i
            for j in range(n):
                x_i += B[i][j] * x[j]

            # Finally, value of the coordinate of vector is x[i] = b_i0 * x_0 + b_i1 * x_1 + ... + b_in * x_n + c_i
            x[i] = x_i + c[i]

        vector_subtract_norm = norm(vector_subtract(x, prev))    # ||x^(k+1) - x^(k)||
        # Getting iteration, x_1, x_2, ..., x_n, ||x^(k+1) - x^(k)|| in a list to use in table
        info_arr.append([iteration] + list(x) + [vector_subtract_norm])

        # Break criteria ||x^(k+1) - x^(k)|| <= eps
        if vector_subtract_norm <= epsilon:

            # Formatting numbers
            for k in range(len(info_arr)):
                info_arr[k] = ["{:.6f}".format(value) if isinstance(value, float) else value for value in info_arr[k]]

            # Making .xlsx table
            df = pandas.DataFrame(info_arr)
            df.to_excel("output.xlsx",
                        header=["Iteration", "$x_1$", "$x_2$", "$x_3$", "$x_4$", "$||x^{(k+1)} - x^{(k)}||$"],
                        index=None)
            return x

        else:
            prev = list(x)
            iteration += 1


x = zeydel_method(B_matrix, c_vector, chebyshev_norm, 0.00001)      # Our solution
numpy_x = numpy.linalg.solve(A_matrix, b_vector)    # Numpy function solution
print("Our solution:", x)
print("Numpy function's solution:", numpy_x)

Ax_vector = numpy.matmul(A_matrix, x)   # A*x
b_minus_Ax_vector = vector_subtract(b_vector, Ax_vector)    # b-A*x
b_minus_Ax_vector = ["{:.6f}".format(coord) for coord in b_minus_Ax_vector]     # Format nicely b-A*x vector
print("b-A*x: ", b_minus_Ax_vector)
# Our solution with starting vector (10,20,30,40)
y = zeydel_method(B_matrix, c_vector, chebyshev_norm, 0.00001, [10, 20, 30, 40])
print("Our solution with (10, 20, 30, 40) starting vector:", y)

\end{verbatim}
\end{document}


